\documentclass{article}
\usepackage[utf8]{inputenc}
\usepackage{graphicx}

\title{Assignment 3: Project check-in}
\author{Kate Plas}
\date{\today}

\begin{document}

\maketitle

\section{Design}

\subsection{Interviews}
\begin{enumerate}
    \item Interview 2 people about journaling and tracking with a focus on your chosen set of themes.
    \begin{itemize}
        \item \textbf{What do you hope to learn from these interviews?}  
        \begin{itemize}
        \item By interviewing potential users, I hope to learn their needs and goals for tracking instrument practice.
        \end{itemize}
        \item \textbf{What questions did you ask?}
        \begin{itemize}
            \item How do you usually keep track of your goals?
            \item What type of metrics do you want to record about each practice session?
            \item What historical practice data would you like to view to observe your progress?
            \item What goals do you focus on in music practice?
        \end{itemize}
        \item \textbf{What did your interview participants tell you? What did you learn from them?}
        \begin{itemize}
            \item One person said they wanted to get a score for how well they played the music by tracking the percent of correct notes.
            \item Another interviewee wanted to be able to record and save their final music piece and retrieve it later.
            \item Both participants stated that it would be beneficial to measure practice time per day. They want to set time frames for when a piece should be proficiently played.
        \end{itemize}
    \end{itemize}
\end{enumerate}

\subsection{Design Requirements}
Create a written list of design goals and requirements for the user interface. You can address some of these in your implementation or leave some as 'future work' outside the scope of the class.

\begin{itemize}
    \item How to concentrate on what the user wants to do immediately (enter a practice session) while being able to view their progress and goals.
    \item Users want to view their progress.
    \item Users should be able to view a historical entry and edit it.
    \item Users need to view the date and time.
    \item Users need to add today’s entry with the length of time practiced, what they practiced on, and which instrument, etc.
\end{itemize}

\subsection{Sketching Design Alternatives}
Choose 3 interesting design challenges to explore.

\begin{itemize}
    \item How to concentrate on what the user wants to do immediately (enter a practice session) while being able to view their progress and goals.
    \item Users want to view their progress.
    \item Users need to add today’s entry with the length of time practiced, what they practiced on, and which instrument, etc.
\end{itemize}

We will follow a protocol for exploring design alternatives called '10-plus-10'. See in-class notes.  
If you are having trouble generating 10-plus-10 sketches, try sketching 10-plus-10 minutes. Sketch design alternatives for 10 minutes, select from these alternatives, and sketch variations on these for 10 minutes.

\subsection{Create a Prototype}
Create a prototype sketch of your envisioned interface (we will discuss and practice in class Week 2).

\subsection{Feedback on Prototype Sketches}
Show your prototype sketches to 2 people (friends, family members, classmates). Record the feedback.

\subsection{Create a User Profile}
Create a user profile for a mock user. This mock user will be the test case for your application. Write a brief description of them and how they would use this application.

\end{document}
